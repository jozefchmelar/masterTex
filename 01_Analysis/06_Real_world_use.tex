\section{Real world use}

\subsection{e-Estonia} 
The Estonian government has been testing blockchain since 2008. Since 2012, blockchain has been in operational use in Estonia’s registries, such as national health, judicial, legislative, security and commercial code systems, with plans to extend its use to other spheres such as personal medicine, cybersecurity and data embassies.

Blockchain technology solves many of the problems that data governance professionals have been trying to solve for years. The technology developed by the Estonians is also being used by NATO, U.S. Department of Defence, as well as European Union information systems to ensure cybersecurity.

In Estonia, patients own their health data and hospitals have made this available online since 2008. Today, over 95\% of the data generated by hospitals and doctors has been digitized, and blockchain technology is used for assuring the integrity of stored electronic medical records as well as system access logs. Using blockchain technology mitigates internal threats to data, making patient’s information more secure. \cite{estionia}


\subsection{Counterfeit drug prevention and detection}

When we get sick, we trust that the doctors have our best interests at heart. We trust that the medicine we are prescribed and that we purchase will make us better instead of making our condition worse. And by and large, they do. However, that’s not always the case. Counterfeit medicine is a global problem that has adversely impacted hundreds of thousands of people across the world particularly in developing countries that do not have strong regulation structures and enforcement. According to the World Health Organization (WHO) estimates, “1 in 10 medical products circulating in low and middle-income countries is either substandard or falsified”, which ranged from flu vaccines to cancer treatment. \cite{fake_drugs}

FarmaTrust has formed a partnership with the Mongolian government to create a one-year pilot project to prevent the production and distribution of counterfeit medicines using blockchain and other emerging technologies. \cite{fake_drugs}

\subsection{Odometer fraud prevention}

The total economic costs of odometer fraud in second-hand cars traded cross-border in the European Union can be estimated to be at least €1.31 billion, with the most probable fraud rate scenario yielding €8.77 billion of economic loss. The main negative impacts of odometer fraud are borne by consumers, as their rights are breached, confidence is lowered, and maintenance and repair expenses are increased. Road safety is also impacted, as cars are not adequately maintained at the right time. \cite{odometer_fraud}

Bosch IoT Lab is developing a blockchain based solution that ensures a car’s mileage data is correct. They also published an app for consumers that enables them to view the mileage history of their car. Users can access an online service to get a digital certificate indicating whether the mileage has been manipulated or not. By using this service, consumers can easily create a digital certificate for their cars and also share the information with other entities in order to create trust in the specific car data. \cite{bosh}

\subsection{Aerospace supply chains}

Thales is a French multinational company that designs and builds electrical systems and provides services for the aerospace, defence, transportation and security markets. They faced a  problem where the Ministry of Defense had aircraft carriers that needed to be returned to the supplier because of faulty electrical systems. The issue was due to counterfeit components. It meant that the originality of the fake components had to be tracked back up a twenty tier supply chain. \cite{thales}

Aircraft are built using thousands of different suppliers, and any of these suppliers could have inserted counterfeited elements. It can cost millions of pounds to identify the origins of a counterfeit piece. Additionally, a grounded aircraft generates high revenue losses and the need to set up alternative solutions. \cite{thales}

The solution developed by Thales and Accenture uses Fabric Hyperledger allowing for easy allocation of permissions to different participants in the network.