\section{Cryptocurrency}
Cryptocurrency is a digital currency in which encryption techniques are used to control the generation of units of currency and verify the transfer of funds, operating independently of one single central unit  \cite{fintech_compendium}. Cryptocurrencies use decentralized control instead of a single authority like a bank. Decentralization is achieved by using distributed ledger technology - a block chain that publicly stores all transactions in it. 

In the case of centralized banking, the supply of currency is in charge of government institutions like the European Central bank. This way the value of a currency can be adjusted by adding or removing money from circulation. This is not possible with decentralized cryptocurrency where governments or corporations can't produce new units. Cryptocurrencies are by design gradually decreasing the production of their currency with predetermined market cap - the maximum amount of currency that will be in circulation. 

We need to distinguish electronic money from digital currency. 
\begin{description}
\item[Electronic money]
we currently have in our bank account can be considered an electronic version of the cash we would otherwise have in our wallets. Whenever a user uses his electronic money to pay for a service online it's the equivalent of paying in cash. \item[Digital currency]  (or virtual) currency is an electronically issued currency the transferability of which into  fiat currency is not guaranteed by the state. Digital currency can be divided into centralized and decentralized. The best example of centralized digital currency is World Of Warcraft Gold  - the currency used in a computer game.
\end{description}
